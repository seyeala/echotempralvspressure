\documentclass[12pt]{article}
\usepackage[margin=1in]{geometry}
\usepackage{array}
\usepackage{longtable}
\usepackage{hyperref}
\usepackage{listings}
\lstset{
  basicstyle=\ttfamily\small,
  frame=single,
  columns=fullflexible,
  keepspaces=true,
  showstringspaces=false
}
\title{Understanding the Echo vs Pressure Address Indexer}
\author{Project Documentation}
\date{\today}

\begin{document}
\maketitle

\section*{Who is this guide for?}
This document is written for readers who have \\textbf{never written code before}. It explains the project step by step, using plain language and many examples.

\section{What the project does}
\begin{itemize}
  \item The tool looks inside a folder on your computer. We call this folder an \emph{address}.
  \item Inside the address, the tool searches for files that end in ``.csv''. These are simple text tables that can be opened in spreadsheet programs.
  \item It tries to match every ``echo'' CSV file with a special partner named \texttt{pressure.csv}.
  \item It produces three helpful files:
  \begin{enumerate}
    \item \textbf{dataset\_index.json}: a summary in JSON format that lists each folder and the echo--pressure pairs it contains.
    \item \textbf{summary.csv}: a spreadsheet-friendly report that shows the size of every CSV, how many rows and columns it has, and any warnings.
    \item \textbf{errors.log}: a plain text file that lists anything unusual (for example, when \texttt{pressure.csv} is missing).
  \end{enumerate}
\end{itemize}

\section{Why this is useful}
Imagine you are handed a hard drive full of experiment results. Each experiment lives in its own folder and should contain many ``echo'' measurements and exactly one pressure reading. Manually checking every folder would take hours. This tool does that checking automatically and produces concise reports that can be reviewed in minutes.

\section{A gentle tour of the main files}
\begin{longtable}{>{\raggedright\arraybackslash}p{0.2\linewidth}p{0.72\linewidth}}
\textbf{File} & \textbf{Role in plain language}\\\hline
\texttt{src/indexer/main.py} & Provides a command-line interface. It reads your instructions (which folder to scan and where to store the results), calls the scanner, and saves the output files.\\
\texttt{src/indexer/scanner.py} & Does the heavy lifting: walks through folders, identifies CSV files, measures their content, and records warnings.\\
\texttt{src/indexer/io\_helpers.py} & Contains small helper functions to read CSV sizes, write files safely, and create folders as needed.\\
\texttt{src/indexer/models.py} & Defines a simple data container called \texttt{EchoEntry}. It holds the name of an echo file and the matching pressure file.\\
\end{longtable}

\section{Step-by-step: running the tool}
Suppose your data is stored in \texttt{/data/experiments}. Here is what happens when you run the project.

\subsection{1. Start the program}
You use Python to call the tool with the folder you want to scan. The command looks like this:

\begin{lstlisting}
python -m indexer.main --address /data/experiments --outdir artifacts
\end{lstlisting}

\textbf{Translation:} ``Python, please run the indexing program. Look at everything inside \texttt{/data/experiments} and put the reports into a folder named \texttt{artifacts}.''

\subsection{2. Prepare the folders}
The program checks that \texttt{/data/experiments} exists. If it cannot find the folder or read it, you receive a clear error message. It also makes sure the output folder (\texttt{artifacts}) exists, creating it if necessary.

\subsection{3. Discover dataset folders}
The scanner walks through the address and every subfolder. A folder counts as a dataset if it contains at least one CSV file. For example, if your files are arranged like this:

\begin{lstlisting}
/data/experiments
├── TrialA
│   ├── echo_day1.csv
│   ├── echo_day2.csv
│   └── pressure.csv
├── TrialB
│   ├── echo01.csv
│   └── pressure.csv
└── TrialC
    └── pressure.csv
\end{lstlisting}

The tool treats \texttt{TrialA}, \texttt{TrialB}, and \texttt{TrialC} as three separate datasets.

\subsection{4. Collect information about each dataset}
For every dataset folder, the scanner:
\begin{enumerate}
  \item Searches for a file named exactly \texttt{pressure.csv} (not case-sensitive).
  \item Counts how many rows and columns each CSV file contains. Empty files are noted.
  \item Measures each file's size in bytes.
  \item Records everything in memory, ready to be written to disk.
\end{enumerate}

If a dataset folder does not contain \texttt{pressure.csv}, the tool still records the echo files but flags the folder with a warning.

\subsection{5. Write the reports}
After all folders have been scanned, three files are produced:

\begin{description}
  \item[dataset\_index.json] Easy to read by computers and humans. It resembles the structure below:
\end{description}

\begin{lstlisting}
{
  "TrialA": [
    {"echo": "echo_day1.csv", "press": "pressure.csv"},
    {"echo": "echo_day2.csv", "press": "pressure.csv"}
  ],
  "TrialB": [
    {"echo": "echo01.csv", "press": "pressure.csv"}
  ],
  "TrialC": []
}
\end{lstlisting}

If two folders share the same name (for example, two different locations both called \texttt{TrialA}), the tool automatically renames the duplicates to \texttt{TrialA\#2}, \texttt{TrialA\#3}, and so on.

\begin{description}
  \item[summary.csv] A spreadsheet-friendly table. Each row contains:
  \begin{itemize}
    \item The folder path.
    \item The name of the echo file (or \texttt{None} if there were only pressure readings).
    \item The number of rows and columns in that file.
    \item The file size in bytes.
    \item Notes about anything unusual, such as missing pressure data or unreadable files.
  \end{itemize}
  \item[errors.log] Contains human-readable warnings. A typical entry might be:
\end{description}

\begin{lstlisting}
pressure.csv not found in /data/experiments/TrialC
\end{lstlisting}

\section{Understanding the safeguards}
\begin{itemize}
  \item \textbf{Graceful error handling:} The tool keeps scanning even if a folder is unreadable or a CSV is broken. It simply adds a warning to \texttt{errors.log}.
  \item \textbf{Safe file writing:} Reports are first written to temporary files and then swapped into place. This avoids partially written files if the process stops unexpectedly.
  \item \textbf{Flexible character support:} CSV files are read using standard UTF-8 encoding, with an automatic fallback to a broader encoding if needed.
\end{itemize}

\section{Trying it yourself without fear}
\subsection{Create a mini playground}
You can test the tool with dummy data. Open a terminal (or command prompt) and run:

\begin{lstlisting}
mkdir -p /tmp/demo/TrialA /tmp/demo/TrialB
printf "time,echo\n0,10\n1,12\n" > /tmp/demo/TrialA/echo_day1.csv
printf "time,pressure\n0,101\n1,100\n" > /tmp/demo/TrialA/pressure.csv
printf "time,echo\n0,8\n" > /tmp/demo/TrialB/echo.csv
# Note: TrialB intentionally misses pressure.csv
\end{lstlisting}

Then run the indexer:

\begin{lstlisting}
python -m indexer.main --address /tmp/demo --outdir /tmp/demo/artifacts
\end{lstlisting}

Open \texttt{/tmp/demo/artifacts/summary.csv} in a spreadsheet application. You will see rows similar to the table below.

\begin{center}
\begin{tabular}{|l|l|r|r|r|l|}
\hline
Folder & Echo file & Rows & Columns & Size (bytes) & Notes \\\hline
/tmp/demo/TrialA & echo\_day1.csv & 2 & 2 & (about 32) &  \\\hline
/tmp/demo/TrialA & echo\_day2.csv & 0 & 0 & (file missing) & empty\_or\_unreadable \\\hline
/tmp/demo/TrialB & echo.csv & 1 & 2 & (about 24) & missing\_pressure\_csv \\\hline
\end{tabular}
\end{center}

The exact sizes will vary, but the structure will match what the tool records.

\section{Where to go next}
\begin{itemize}
  \item Explore the \texttt{README.md} file for installation tips and advanced usage.
  \item Browse the \texttt{tests/} directory to see automated checks that guarantee the tool behaves as described.
  \item If you grow curious about the code, start with \texttt{src/indexer/main.py}; it reads almost like a recipe of the steps described above.
\end{itemize}

\section*{Glossary}
\begin{description}
  \item[CSV] ``Comma-separated values,'' a plain-text spreadsheet format.
  \item[JSON] ``JavaScript Object Notation,'' a text format often used for structured data.
  \item[Command line] A place where you type instructions instead of clicking buttons. On Windows it is ``Command Prompt'' or ``PowerShell''; on macOS and Linux it is ``Terminal''.
\end{description}

\end{document}
